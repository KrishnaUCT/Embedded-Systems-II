\documentclass[conference]{IEEEtran}
\IEEEoverridecommandlockouts
% The preceding line is only needed to identify funding in the first footnote. If that is unneeded, please comment it out.
\usepackage{cite}
\usepackage{amsmath,amssymb,amsfonts}
\usepackage{algorithmic}
\usepackage{graphicx}
\usepackage{textcomp}
\usepackage{xcolor}
\def\BibTeX{{\rm B\kern-.05em{\sc i\kern-.025em b}\kern-.08em
    T\kern-.1667em\lower.7ex\hbox{E}\kern-.125emX}}
\begin{document}

\title{EEE3096S Prac Titlesss}

\author{\IEEEauthorblockN{1\textsuperscript{st} Student Name}
\IEEEauthorblockA{\textit{Student Number}}
\and
\IEEEauthorblockN{2\textsuperscript{nd} Student Name}
\IEEEauthorblockA{\textit{Student Number}}
}

\maketitle

\begin{abstract}
This document is a model and instructions for \LaTeX.
This and the IEEEtran.cls file define the components of your paper [title, text, heads, etc.]. *CRITICAL: Do Not Use Symbols, Special Characters, Footnotes, 
or Math in Paper Title or Abstract. boooo
\end{abstract}


\section{Introduction}
This practical focuses on Embedded ARM Assembly Programming. Assembly code is written to interface the STM32 microcontroller board to perform basic operations. The key objectives involves knowing how to manipulate LEDs based on user input from pushbuttons on the board, using Assembly language. In the practical assembly code is written to manipulate the time delay loops and the pattern displayed by the LEDs based on what pushbuttons are pressed.


\section{Methodology}

The template for integration of the STM32F0 with assembly was provided; the necessary C code and boiler plate for the assembly was used as a starting point. A flowchart was created to establish the desired states and workflow.

To implement the solution, the reference manual and Assembly resources provided were utilized along with sample code found through research of similar projects. To validate the solution, the website generated by the teaching team is very useful; many bugs can be identified.

The assembly code to write a binary number to the LEDs was provided. Each requirement was developed one at a time after that, 

\subsection{Figures and Tables}

\begin{figure}[htbp]
\centerline{\includegraphics{fig1.png}}
\caption{Example of a figure caption.}
\label{fig}
\end{figure}

\section{Results and Discussion}



\section{Conclusion}




\section*{Acknowledgment and AI Clause}

The template and boilerplate code was provided by the EEE3096S teaching team. ChatGPT and Claude were used to understand the provided template code. In addition, it was used for understanding the concepts of the Assembly programming language, including syntax, logic flow and control as well as assistance with debugging. No generated code was copied directly.

\begin{thebibliography}{00}
\bibitem{b1}
Mikrocontroller.net,
ARM-ASM-Tutorial,
October 2019,
Available at: https://www.mikrocontroller.net/attachment/431716/ArmAssemblerTutorial.pdf,
accessed August 26, 2025.

\end{thebibliography}

\section*{Appendix}
The code used for this profiling was modified from
github.com/EEE3096S-UCT/EEE3096S-2025; the modified
code is available at github.com/KrishnaUCT/Embedded-
Systems-II and the assembly.s file is attached for reference

\vspace{12pt}
\end{document}

